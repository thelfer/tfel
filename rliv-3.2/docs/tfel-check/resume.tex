%%%%%%%%%%%%%%%%%%%%%%%%%%%%%%%%%%%%%%%%%%%%%%%%%%%%%%%%%%%%%%%%%%%%%%%%%%%%%%%
%%	Fichier	   : resume.tex
%%	Auteur     : helfer@tfel001.cad.cea.fr
%%	Date       : 01 Feb 2008
%%%%%%%%%%%%%%%%%%%%%%%%%%%%%%%%%%%%%%%%%%%%%%%%%%%%%%%%%%%%%%%%%%%%%%%%%%%%%%%

Les applications \tfel{} évoluent. Cette évolution est liée d'une part
au développement des applications elles-mêmes mais également au développement
de l'architecture \tfel{} et au renouvellement des différents pré-recquis
qu'elle utilise (compilateurs,support du format MED,\castem{},
{\tt py\castem{}}, etc.).

Comme tout développement informatique, ces évolutions peuvent conduire à des
modifications des résultats, même si les modèles physiques restent
{\em inchangés}. Ces modifications peuvent avoir plusieurs origines, nous pensons
notamment aux modifications des algorithmes numériques utilisés.

Cette note présente un outil, nommé {\tt tfel-check}, permettant
de contrôler de manière {\em automatisée} sur un ensemble de calculs de référence
que les évolutions des applications sont {\em stables}, c'est à dire
que les résultats ne sont que peu \og~perturbés~\fg, au sens d'une erreur
définie par le développeur. Cet outil s'intègre et renforce
la stratégie \og~qualité~\fg mise en place dans la plate-forme \tfel{}.

Cet outil a initialement été développé pour les besoins de l'application
\celaeno{}.

%%% Local Variables: 
%%% mode: latex
%%% TeX-master: t
%%% End: 
